\chapter{Технологическая часть}

В данном разделе описывается техническая реализация HTTP-сервера: выбор языка программирования,
поддерживаемые типы запросов, а также ключевые особенности реализации,
обеспечивающие безопасность, производительность и корректность работы.

\section{Выбор языка программирования}

Для реализации сервера был выбран язык \textbf{C} по следующим причинам:
\begin{itemize}
    \item[---] Прямой доступ к системным вызовам POSIX (\textit{socket}, \textit{poll}, \textit{sendfile}, \textit{open} и др.), что необходимо для точного контроля над сетевым и файловым вводом-выводом.
    \item[---] Минимальные накладные расходы: отсутствие автоматической сборки мусора и runtime-библиотек позволяет достичь высокой производительности и предсказуемого поведения.
    \item[---] Портативность: код компилируется на любой POSIX-совместимой системе (Linux, BSD, macOS) с помощью стандартного компилятора \textit{gcc}.
\end{itemize}

Для сборки используется система \textit{make}, что обеспечивает кроссплатформенность и воспроизводимость сборки.

\section{Поддерживаемые HTTP-методы и статусы}

Сервер поддерживает два метода протокола HTTP/1.1:
\begin{itemize}
    \item[---] \textit{GET} -- запрашивает полное содержимое ресурса (заголовки с телом).
    \item[---] \textit{HEAD} -- запрашивает только метаданные (заголовки без тела).
\end{itemize}

При получении любого другого метода, сервер возвращает код состояния: 405 (\textit{Method Not Allowed}).

Поддерживаемые коды ответа:
\begin{itemize}
    \item[---] 200 (\textit{OK}) -- успешная обработка запроса;
    \item[---] 400 (\textit{Bad Request}) -- ошибка формата запроса (слишком длинный путь и т.п.);
    \item[---] 403 (\textit{Forbidden}) -- доступ запрещён;
    \item[---] 404 (\textit{Not Found}) -- файл не найден;
    \item[---] 413 (\textit{Payload Too Large}) -- запрошенный файл превышает лимит в 128~МБ;
    \item[---] 500 (\textit{Internal Server Error}) -- ошибка формирования заголовка.
\end{itemize}

\section{Поддерживаемые типы файлов}

Сервер автоматически определяет MIME-тип отдаваемого файла по расширению и указывает его в заголовке \textit{Content-Type}. Поддерживаются следующие типы:

\begin{itemize}
    \item[---] \textbf{Веб-контент}: \textit{.html}, \textit{.htm}, \textit{.css}, \textit{.js}, \textit{.json};
    \item[---] \textbf{Изображения}: \textit{.png}, \textit{.jpg}, \textit{.jpeg}, \textit{.gif}, \textit{.ico}, \textit{.webp};
    \item[---] \textbf{Видео}: \textit{.mp4}, \textit{.webm}, \textit{.ogg}, \textit{.ogv}, \textit{.avi}, \textit{.mov}, \textit{.mkv};
    \item[---] \textbf{Документы}: \textit{.txt}, \textit{.pdf};
    \item[---] \textbf{Прочее}: все остальные файлы отдаются с типом \textit{application/octet-stream}.
\end{itemize}

Это позволяет корректно отображать веб-страницы с изображениями, стилями, скриптами и мультимедиа.

\subsection{Обработка больших файлов}

Файлы размером до 128~МБ передаются по частям с использованием системного вызова \textit{sendfile()},
который выполняет копирование данных напрямую из ядра (без копирования в пользовательское пространство --
zero-copy). Отправка производится асинхронно: после каждой попытки вызывается \textit{poll()},
что предотвращает блокировку потока.

\subsection{Безопасность}

Реализована защита от атак типа \textit{Path Traversal}:
\begin{itemize}
    \item[---] Все входящие пути нормализуются через \textit{realpath()}, что устраняет последовательности \textit{../} и символические ссылки.
    \item[---] Проверяется, что результирующий путь находится строго внутри корневой директории.
    \item[---] Запросы к директориям (без \textit{index.html}) возвращают 403 (\textit{Forbidden}).
\end{itemize}

\subsection{Мультиплексирование и пул потоков}

Архитектура основана на комбинации:
\begin{itemize}
    \item[---] \textbf{Пула потоков} -- фиксированное число рабочих потоков (настраивается при запуске).
    \item[---] \textbf{Мультиплексирования через \textit{poll()}} -- каждый поток одновременно обслуживает до 1024 соединений.
\end{itemize}

Новые соединения распределяются между потоками по алгоритму \textit{round-robin}
через анонимные каналы (pipes), что исключает конкуренцию за мьютексы в основном пути данных.