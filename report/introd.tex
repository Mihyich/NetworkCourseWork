\chapter*{ВВЕДЕНИЕ}
\addcontentsline{toc}{chapter}{ВВЕДЕНИЕ}

\textbf{Целью} курсовой работы является разработка HTTP-сервера для отдачи статического
содержимого, построенного на архитектуре пула потоков с мультиплексированием сетевых
соединений с использованием системного вызова \texttt{poll()}.

Задачи:
\begin{itemize}
	\item[---] Предусмотреть поддержку запросов GET и HEAD, поддержку статусов 200, 403, 404.
	\item[---] Предусмотреть возможность ответа сервера на неподдерживаемые запросы статусом 405.
	\item[---] Обеспечить корректную передачу файлов размером до 128 Мбайт.
	\item[---] Реализовать мультиплексирование -- каждый процесс или поток должен отдавать данные по нескольким сетевым соединениям.
	\item[---] Сервер по умолчанию должен возвращать HTML-страницу на выбранную тему с CSS-стилем.
	\item[---] Реализовать запись информации о событиях в журнал (лог).
	\item[---] Учесть минимальные требования к безопасности серверов статического содержимого.
\end{itemize}