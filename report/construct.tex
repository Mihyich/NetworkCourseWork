\chapter{Конструкторская часть}

В данном разделе представлена архитектура HTTP-сервера,
описаны ключевые компоненты и алгоритмы их взаимодействия.

\section{Общая архитектура сервера}

Сервер состоит из следующих модулей:

\begin{enumerate}[label=\arabic*), labelsep=0.5em]
    \item \textbf{Главный поток} -- создаёт слушающий сокет, запускает пул рабочих потоков и принимает новые соединения через \texttt{accept()}.
    \item \textbf{Пул рабочих потоков} -- фиксированное количество потоков, каждый из которых управляет множеством клиентских соединений.
    \item \textbf{Каналы (pipes)} -- используются для передачи новых сокетов от главного потока к рабочим.
    \item \textbf{Модуль HTTP} -- парсинг запросов, формирование ответов.
    \item \textbf{Модуль безопасности} -- проверка пути через \textit{is\_path\_safe}.
    \item \textbf{Логгер} -- потокобезопасная запись событий.
\end{enumerate}

\section{Жизненный цикл клиентского соединения}

Каждое соединение проходит через конечный автомат с четырьмя состояниями (рис.~\ref{chart:connection}):

\begin{enumerate}
    \item \textit{CONN\_READING} -- ожидание и чтение HTTP-запроса;
    \item \textit{CONN\_SENDING\_HEADER} -- отправка заголовков ответа;
    \item \textit{CONN\_SENDING\_BODY} -- пошаговая отправка тела файла через \textit{sendfile()};
    \item \textit{CONN\_DONE} -- завершение, закрытие сокета и файла.
\end{enumerate}

Переходы между состояниями управляются событиями \textit{POLLIN} и \textit{POLLOUT},
возвращаемыми вызовом \textit{poll()}.

\begin{figure}[h]
    \centering
    \includegraphics[width=0.8\textwidth]{img/connection/chart.pdf}
    \caption{Диаграмма состояний жизненного цикла клиентского соединения}
    \label{chart:connection}
\end{figure}

\FloatBarrier

\section{Алгоритм главного потока}

Главный поток выполняет следующие шаги:
\begin{enumerate}[label=\arabic*), labelsep=0.5em]
    \item Создать слушающий сокет.
    \item Запустить пул рабочих потоков.
    \item Запустить цикл, пока работа сервера не завершена:
    \begin{enumerate}[label=3.\arabic*), labelsep=0.5em]
        \item Если новое соединение установлено:
        \begin{itemize}
            \item[---] Получить IP и порт клиента;
            \item[---] Назначить соединение рабочему потоку.
        \end{itemize}
        \item Конец цикла.
    \end{enumerate}
    \item Остановить пул рабочих потоков.
    \item Закрыть слушающий сокет.
\end{enumerate}

\section{Алгоритм рабочего потока}

Рабочий поток реализует цикл обработки событий:
\begin{enumerate}[label=\arabic*), labelsep=0.5em]
    \item Создать массив соединений и pipe для приема новых сокетов.
    \item Цикл до завершения работы потока:
    \begin{enumerate}[label=2.\arabic*), labelsep=0.5em]
        \item Сформировать набор дескрипторов для \textit{poll()}.
        \item Вызвать \textit{poll()}.
        \item Если пришли новые соединения (события на pipe) -- добавить их в массив в состоянии \textit{CONN\_READING}.
        \item Для каждого активного сокета:
        \begin{itemize}
            \item[---] При \textit{POLLIN} и состоянии \textit{READING} -- прочитать и проанализировать запрос, при успехе перейти к подготовке ответа.
            \item[---] При \textit{POLLOUT} и состоянии \textit{SENDING\_HEADER} -- отправить заголовок, при завершении перейти к отправке тела.
            \item[---] При \textit{POLLOUT} и состоянии \textit{SENDING\_BODY} -- отправить фрагмент файла.
            \item[---] При ошибках или завершении -- перевести соединение состояние в \textit{CONN\_DONE}.
        \end{itemize}
        \item Удалить все соединения в состоянии \textit{CONN\_DONE}.
    \end{enumerate}
    \item При завершении -- освободить все ресурсы.
\end{enumerate}

\section{Механизм передачи соединений через pipe}

Для назначения нового соединения рабочему потоку используется \textit{round-robin}
распределение:
\begin{enumerate}
    \item Главный поток формирует структуру \textit{conn\_msg = \{fd, ip, port\}}.
    \item Выбирается следующий рабочий поток по индексу.
    \item Структура записывается в pipe этого потока.
    \item Рабочий поток получает событие \textit{POLLIN} на своем pipe и добавляет сокет в свой набор.
\end{enumerate}

Этот механизм обеспечивает потокобезопасную и неблокирующую передачу дескрипторов
без использования мьютексов в основном пути.

