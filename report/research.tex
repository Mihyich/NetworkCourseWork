\chapter{Исследовательская часть}

В данном разделе представлены результаты нагрузочного тестирования разработанного HTTP-сервера.
Целью исследования является оценка:
\begin{itemize}
    \item[---] максимального количества одновременно обслуживаемых соединений;
    \item[---] скорости отдачи данных по одному и множеству соединений;
    \item[---] общей пропускной способности сервера.
\end{itemize}

\section{Характеристики оборудования}

Характеристики оборудования, на котором проводилось исследование приведены ниже:
\begin{enumerate}
	\item[---] \textbf{ОС}: Ubuntu 22.04.1 LTS Release: 22.04,
	\item[---] \textbf{ЦП}: AMD Ryzen 7 4800H with Radeon Graphics, 2900 МГц, ядер: 8,
	\item[---] \textbf{ОЗУ}: 16 Гб, 3200 МГц,
	\item[---] \textbf{ПЗУ}: INTEL SSDPEKNW512G8H 476,9Гб.
\end{enumerate}

\section{Результаты тестирования}

Для генерации нагрузки использовался инструмент \texttt{wrk} версии 4.2.0,
а также \texttt{curl} версии 8.5.0.  
Тестовые файлы:
\begin{itemize}
    \item[---] \textit{index.html} -- HTML-файл (5.3 КБ) для тестирования максимального количества сетевых соединений;
    \item[---] \textit{dataset.bin} -- бинарный файл размером 100 МБ для тестирования отдачи данных по каждому сетевому соединению и совокупному.
\end{itemize}

\subsection{Тестирование максимального количества обслуживаемых сетевых соединений}

Для тестирования максимального количества обслуживаемых сетевых соединений использовалась утилита \texttt{wrk}.
Сервер был запущен с помощью команды:
\begin{equation*}
	\textit{./app ./htdocs 8080 threads},\text{ где }\textit{threads} = 1, 2, 4, 8, 16.
\end{equation*}
Для каждого запуска с разным количеством рабочих потоков в пуле, выполнялся запуск утилиты \texttt{wrk} следующим образом:
\begin{equation*}
	\textit{wrk -t\textbf{threads} -c\textbf{connections} -d30s http://localhost:8080/index.html},
\end{equation*}
где \textbf{threads} соответствует количеству рабочих потоков в текущем запуске сервера,
а \textbf{connections} $= 10, 50, 100, 200, 500, 1000$ для каждого количества рабочих потоков
в пуле, кроме случая, когда \textbf{threads} $= 16$, в этом случае \textbf{connections} перебираются так:
$16, 50, 100, 200, 500, 1000$.

Однако, в процессе тестирования возникла ошибка сервера: \textit{Too many open files} --
которая обозначает, что процесс превысил лимит на количество открытых файловых дескрипторов.
Для решения этой проблемы, временно был вручную увеличен лимит с помощью~\cite{ulimit}:
\begin{equation*}
	\text{ulimit -n 4096}.
\end{equation*}

Полученные результаты записаны в таблицах~\ref{tab:ex1}~--~\ref{tab:ex2}. На графиках~\ref{graph:throughput_vs_connections}~-~
\ref{graph:latency_vs_connections} представлены зависимости количества обрабатываемых запросов в секунду,
совокупного объема переданных данных, средней задержки сервера от числа одновременных соединений при тестовом сценарии с легковесным
файлом (\textit{index.html}) соответственно.

\begin{sidewaystable}
\centering
\raggedright
\caption{Результаты тестирования максимального количества обслуживаемых сетевых соединений (часть 1)}
\label{tab:ex1}
\begin{tabular}{|p{2.9cm}|p{3.1cm}|p{3.6cm}|p{7.2cm}|p{6.1cm}|}
\hline
\textbf{Количество потоков} & \textbf{Количество соединений} & \textbf{Средняя задержка}, (мкс) & \textbf{Количество обработанных запросов}, ($\frac{10^3}{\text{с}}$) & \textbf{Объем отправленных данных}, (Мб) \\
\hline

1 & 10 & 290.28 & 20.51 & 104.47 \\
\hline
1 & 50 & 1\,450 & 21.09 & 107.41 \\
\hline
1 & 100 & 2\,820 & 20.72 & 105.52 \\
\hline
1 & 200 & 5\,320 & 20.17 & 102.66 \\
\hline
1 & 500 & 1\,364 & 19.43 & 98.80 \\
\hline
1 & 1000 & 2\,585 & 19.78 & 100.65 \\
\hline

2 & 10 & 165.39 & 16.87 & 171.84 \\
\hline
2 & 50 & 686.94 & 19.24 & 195.92 \\
\hline
2 & 100 & 1\,390 & 19.04 & 193.85 \\
\hline
2 & 200 & 2\,450 & 19.56 & 199.09 \\
\hline
2 & 500 & 6\,090 & 18.97 & 192.84 \\
\hline
2 & 1000 & 1\,243 & 18.50 & 187.94 \\
\hline

4 & 10 & 123.79 & 8.41 & 171.35 \\
\hline
4 & 50 & 461.92 & 14.63 & 297.63 \\
\hline
4 & 100 & 743.84 & 16.91 & 344.49 \\
\hline
4 & 200 & 1\,330 & 17.26 & 351.47 \\
\hline
4 & 500 & 3\,170 & 17.03 & 346.44 \\
\hline
4 & 1000 & 6\,550 & 16.35 & 332.04 \\
\hline

8 & 10 & 123.79 & 8.41 & 171.35 \\
\hline
8 & 50 & 461.92 & 14.63 & 297.63 \\
\hline
8 & 100 & 743.84 & 16.91 & 344.49 \\
\hline
8 & 200 & 1\,330 & 17.26 & 351.47 \\
\hline
8 & 500 & 3\,170 & 17.03 & 346.44 \\
\hline
8 & 1000 & 6\,550 & 16.35 & 332.04 \\
\hline

\end{tabular}

\end{sidewaystable}

\FloatBarrier

\begin{sidewaystable}
\centering
\raggedright
\caption{Результаты тестирования максимального количества обслуживаемых сетевых соединений (часть 2)}
\label{tab:ex2}
\begin{tabular}{|p{2.9cm}|p{3.1cm}|p{3.6cm}|p{7.2cm}|p{6.1cm}|}
\hline
\textbf{Количество потоков} & \textbf{Количество соединений} & \textbf{Средняя задержка}, (мкс) & \textbf{Количество обработанных запросов}, ($\frac{10^3}{\text{с}}$) & \textbf{Объем отправленных данных}, (Мб) \\
\hline

8 & 10 & 72.94 & 4.17 & 169.68 \\
\hline
8 & 50 & 457.15 & 8.69 & 353.95 \\
\hline
8 & 100 & 663.39 & 10.56 & 430.19 \\
\hline
8 & 200 & 1\,080 & 12.64 & 514.78 \\
\hline
8 & 500 & 2\,410 & 12.65 & 514.75 \\
\hline
8 & 1000 & 4\,730 & 12.30 & 499.79 \\
\hline

16 & 16 & 160.92 & 3.98 & 323.19 \\
\hline
16 & 50 & 408.43 & 5.33 & 433.69 \\
\hline
16 & 100 & 761.43 & 5.95 & 484.56 \\
\hline
16 & 200 & 1\,510 & 6.07 & 494.44 \\
\hline
16 & 500 & 3\,790 & 5.92 & 481.95 \\
\hline
16 & 1000 & 7\,370 & 5.65 & 459.24 \\
\hline

\end{tabular}

\end{sidewaystable}

\FloatBarrier

\begin{figure}[h]
    \centering
    \includegraphics[width=0.9295\textwidth]{data/test1/throughput_vs_connections.pdf}
    \caption{Зависимость количества обрабатываемых запросов в секунду от числа одновременных соединений при отдаче index.html}
    \label{graph:throughput_vs_connections}
\end{figure}

\FloatBarrier

\begin{figure}[h]
    \centering
    \includegraphics[width=0.9295\textwidth]{data/test1/transfer_connections.pdf}
    \caption{Зависимость совокупного объема переданных данных от числа одновременных соединений при отдаче index.html}
    \label{graph:transfer_connections}
\end{figure}

\FloatBarrier

\begin{figure}[h]
    \centering
    \includegraphics[width=0.9295\textwidth]{data/test1/latency_connections.pdf}
    \caption{Зависимость средней задержки сервера от числа одновременных соединений при отдаче index.html}
    \label{graph:latency_vs_connections}
\end{figure}

\FloatBarrier

\subsection{Тест 2: Отдача большого файла}

Для оценки передачи объемных данных сервер был запущен с оптимальным количеством потоков -- 8:
\begin{equation*}
	\textit{./app ./htdocs 8080 8}.
\end{equation*}

Для тестирования отдачи данных по одному сетевому соединению использовался файл
\textit{dataset.bin} размером 100 Мб с помощью команды:
\begin{equation*}
	\text{time (curl -o /dev/null http://localhost:8080/resources/test/dataset.bin \& wait)},
\end{equation*}
которая была последовательно выполнена 7 раз (таблица~\ref{tab:ex3}).

\begin{table}[h]
	\centering
	\caption{Тестирование отдачи данных по одному сетевому соединению}
	\label{tab:ex3}
	\begin{tabular}{|p{2.6cm}|p{5cm}|p{8cm}|}
		\hline
		\textbf{№ запуска} & \textbf{Время}, (мс) & \textbf{Пропускная способность}, (Мб/с) \\
		\hline
		1 & 56 & 2\,444 \\
		\hline
		2 & 52 & 2\,676 \\
		\hline
		3 & 46 & 3\,236 \\
		\hline
		4 & 54 & 2\,547 \\
		\hline
		5 & 46 & 3\,221 \\
		\hline
		6 & 46 & 3\,237 \\
		\hline
		7 & 46 & 3\,227 \\
		\hline
		\textbf{Среднее} & \textbf{49.42} & \textbf{2\,941.14} \\
		\hline
	\end{tabular}
\end{table}

\FloatBarrier

На рисунке~\ref{graph:one_conn} представлен график зависимости пропускной способности (Мб/с) сервера
от номера запуска команды \textit{curl}, а также изображена средняя пропускная способность сервера при отдаче
файла объемом 100 Мб по одному сетевому соединению.

\begin{figure}[h]
    \centering
    \includegraphics[width=\textwidth]{data/test2/one_conn.pdf}
    \caption{Зависимость пропускной способности сервера от номера запуска команды \textit{curl} при передаче 100 МБ по одному сетевому соединению}
    \label{graph:one_conn}
\end{figure}

\FloatBarrier

На рисунке~\ref{graph:one_conn_time} представлен график зависимости времени выполнения команды \textit{curl} (мс)
от номера ее запуска, а также изображено среднее время выполнения этой команды при отдаче
файла объемом 100 Мб по одному сетевому соединению.

\begin{figure}[h]
    \centering
    \includegraphics[width=\textwidth]{data/test2/one_conn_time.pdf}
    \caption{Зависимость времени выполнения команды \textit{curl} от номера ее запуска при передаче 100 МБ по одному сетевому соединению}
    \label{graph:one_conn_time}
\end{figure}

\FloatBarrier

Для тестирования отдачи данных по нескольким параллельным сетевым соединениям сервер также
был запущен с оптимальным количеством рабочих поток в пуле (8), а для создания параллельных
соединений использовался \textit{bash}-скрипт~\ref{lst:data/test2/test.sh}~\cite{bash}.

\begin{table}[h]
	\centering
	\caption{Тестирование отдачи данных по нескольким параллельным сетевым соединениям}
	\label{tab:ex4}
	\begin{tabular}{|p{3cm}|p{5cm}|p{5cm}|p{2.2cm}|}
		\hline
		\textbf{Количество сетевых соединений} & \textbf{Прорускная способность на соединение}, (Мб/с) & \textbf{Совокупная пропускная способность}, (Мб/с) & \textbf{Время}, (мс) \\
		\hline
		1 & 5\,108.11 & 5\,108.11 & 20 \\
		\hline
		2 & 6\,358.69 & 12\,717.38 & 16 \\
		\hline
		4 & 6\,201.96 & 24\,807.85 & 16 \\
		\hline
		8 & 7\,260.14 & 58\,081.14 & 14 \\
		\hline
		16 & 5\,947.57 & 95\,161.15 & 17 \\
		\hline
		32 & 3\,842.48 & 122\,959.52 & 26 \\
		\hline
		64 & 2\,252.20 & 144\,141.37 & 44 \\
		\hline
		128 & 1\,198.97 & 153\,468.88 & 83 \\
		\hline
		256 & 575.49 & 147\,327.97 & 174 \\
		\hline
		512 & 260.38 & 133\,316.38 & 384 \\
		\hline
		1\,024 & 145.50 & 148\,992.23 & 687 \\
		\hline
		\textbf{Среднее} & \textbf{3559.22} & \textbf{95\,098.36} & \textbf{134.63} \\
		\hline
	\end{tabular}
\end{table}

\FloatBarrier

По полученным данным были построены графики зависимости средней скорости отдачи данных на соединение,
совокупной пропускной способности сервера и времени передачи от количества сетевых
соединений при отдаче файла объемом 100 Мб по нескольким параллельным сетевым соединениям
соответственно (рисунки~\ref{graph:ex2}~--~\ref{graph:time_power}).

\begin{figure}[h]
    \centering
    \includegraphics[width=\textwidth]{data/test2/per_power.pdf}
    \caption{Зависимость средней скорости отдачи данных на соединение от количества сетевых соединений}
    \label{graph:ex2}
\end{figure}

\FloatBarrier

\begin{figure}[h]
    \centering
    \includegraphics[width=\textwidth]{data/test2/aver_power.pdf}
    \caption{Зависимость совокупной пропускной способности сервера от количества сетевых соединений}
    \label{graph:ex3}
\end{figure}

\FloatBarrier

\begin{figure}[h]
    \centering
    \includegraphics[width=\textwidth]{data/test2/time_power.pdf}
    \caption{Зависимость времени передачи от количества сетевых соединений}
    \label{graph:time_power}
\end{figure}

\FloatBarrier

\section{Вывод}

На основе проведенного нагрузочного тестирования сервера сделаны следующие выводы:
\begin{itemize}
	\item[---] При увеличении количества рабочих потоков с 1 до 8 наблюдается рост совокупной
	пропускной способности (в Мб/с) и числа обработанных запросов в секунду.
	\item[---] Однако дальнейшее увеличение числа потоков до 16 приводит к снижению производительности.
	\item[---] Сервер демонстрирует максимальную эффективность при 8 потоках, что соответствует числу физических ядер ЦП.
	\item[---] При передаче 100 Мб по одному соединению средняя скорость -- 2941 Мб/с,
	а в пике -- до 3237 Мб/с. Такие значения близки к пропускной способности ПЗУ
	(INTEL SSDPEKNW512G8: до ~2000-3500 Мб/с при последовательном чтении).
	\item[---] При параллельной передаче 100-Мб файла по множеству соединений,
	совокупная пропускная способность растет до 153 Гб/с при 128 соединениях, но
	начиная с 256 соединений начинает снижаться (147 Гб/с при 256, 133 Гб/с при 512).
\end{itemize}


